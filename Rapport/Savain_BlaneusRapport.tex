\documentclass[12pt]{article}
\usepackage[french]{babel}
\usepackage{graphicx}
\usepackage[utf8]{inputenc}
\begin{document}{}
\title{OsuFake}
\author{Blaneus Savain L1-B1}
\date{}
\maketitle

\tableofcontents
\section{Introduction}
J'ai décidé comme projet de réaliser entièrement un jeu auquel j'aime beaucoup Osu (avec tous ces modes). \newline 
Osu est un jeu de rythme ayant 3 modes principale :
\begin{itemize}
\item Osu standard 
\item Osu Mania
\item Osu ctb 
\end{itemize}

\vspace*{3cm}
\section{Fonctionnement du jeux/Gameplay}
\begin{center}
Le gameplay du jeux, j'ai décidé de commencer par le mode principale d'Osu (le jeux auquel je m'inspire) car c'est le mode ou je me disais que c’était le plus facile à faire 

\end{center}
Le gameplay est simple des cercles vont apparaitre sur l'écran pendant quelques secondes pendant ce laps de temps le joueur va devoir appuyer sur le cercle soit en utilisant la souris soit en utilisant une touche de clavier (je vous invite à regarder des gameplay sur YouTube si vous avez du mal à visualiser.) pour nous j'ai donc décidé d'utiliser juste la souris, car ça risque de me compliquer la tache un peu pour rien. 


\section{Difficulté rencontré}
\begin{itemize}
\item J'ai rencontré quelques problèmes, par exemple la syntaxe orientée objet qui m'a beaucoup perturbé j'avais du mal à comprendre et l'utiliser, mais je me suis forcé à faire, car on ne peut pas passer à côté de ça.
\item À propos du mode standard, je me disais que c'était très simple que la seule chose à faire est de mettre un timer qui déplacera l'image toute les tant de seconde, mais en essayant, je me suis rendu compte que c'étais un peu plus compliqué que ça.
\end{itemize} 
\section{Conclusion}
Ce projet m'a aidé à mieux comprendre plusieurs principe comme celui du développement oriente objet.
l'importance de commenté son code.


\end{document}